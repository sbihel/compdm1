\documentclass[a4paper,11pt,english]{article}

\usepackage[T1]{fontenc}
\usepackage[utf8]{inputenc}
\usepackage{babel}
\usepackage{lmodern}
\usepackage{inconsolata} % default typewriter kills me
\usepackage{listings}
\usepackage[htt]{hyphenat} % hyphen with tt
\usepackage{enumitem}
\usepackage{fullpage}

\setlist[enumerate]{leftmargin=*}

\setlength{\parindent}{0cm}

\title{COMP --- TP2\\VSL+ Compiler}
\author{Simon Bihel \and Lesly-Ann Daniel}

\begin{document}

\maketitle

\section{Introduction}
\textit{Objectives described with our own words}

The goal of the project was to write a code generator.
It takes a tree built by an already existing parser.
The compiler takes a VSL+ program and returns a MIPS program.

\section{Work Organization}
\textit{work splitting, code organization, tests}
The tree parser contains 3 files, one main tree parser, one that contains functions to generate three-address code instructions and one file for error detection and handling.
For the development phase, each feature was tested incrementally.
At the beginning we would change the starting rule to test newly added instructions because blocks or functions were not supported yet.

\section{Work Done}
\textit{what has been done}

\section{Tests}
\textit{tests done}
We wrote a bash script that executes our compiler on tests and compares the text output to what is expected.

\section{Possible Optimizations}
\textit{what we've talked about with luc}
During a TD class we discussed optimizations that could be implemented in this project by adding an optimization phase in the tree parser.
First there is the decreasing ordering of variables declarations.
Then there is the loop unfolding that would allow to cut out some variables re-declarations.

\section{Conclusion}
\textit{difficulties encountered, what we've learned, ...}
There was proposed tests for array pointer parameters.
While we were told that we did not have to pass them, interesting questions emerged.
From available types of objects, it does not seem feasible.

\end{document}
