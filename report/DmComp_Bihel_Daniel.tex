%%%%%%%%%%%%%%%%%%%%%%%%%%%%%%%%%%%%%%%%%
% Short Sectioned Assignment
% LaTeX Template
% Version 1.0 (5/5/12)
%
% This template has been downloaded from:
% http://www.LaTeXTemplates.com
%
% Original author:
% Frits Wenneker (http://www.howtotex.com)
%
% License:
% CC BY-NC-SA 3.0 (http://creativecommons.org/licenses/by-nc-sa/3.0/)
%
%%%%%%%%%%%%%%%%%%%%%%%%%%%%%%%%%%%%%%%%%

%----------------------------------------------------------------------------------------
%	PACKAGES AND OTHER DOCUMENT CONFIGURATIONS
%----------------------------------------------------------------------------------------

\documentclass[paper=a4, fontsize=11pt]{scrartcl} % A4 paper and 11pt font size

\usepackage[utf8]{inputenc}    
\usepackage[T1]{fontenc}
%\usepackage{fourier} % Use the Adobe Utopia font for the document
\usepackage[french]{babel} % English language/hyphenation
\usepackage{amsmath,amsfonts,amsthm} % Math packages

\usepackage{color}
\usepackage{xcolor}
\usepackage{listings}
\usepackage{lstlinebgrd}

\usepackage{expl3,xparse}

\ExplSyntaxOn
\NewDocumentCommand \lstcolorlines { O{green} m }
{
 \clist_if_in:nVT { #2 } { \the\value{lstnumber} }{ \color{#1} }
}
\ExplSyntaxOff

\usepackage{caption}
\DeclareCaptionFont{white}{\color{white}}
\DeclareCaptionFormat{listing}{\colorbox{gray}{\parbox{\textwidth}{#1#2#3}}}
\captionsetup[lstlisting]{format=listing,labelfont=white,textfont=white}

\usepackage{sectsty} % Allows customizing section commands
\allsectionsfont{\centering \normalfont \scshape} % Make all sections centered, the default font and small caps

\usepackage{fancyhdr} % Custom headers and footers
\pagestyle{fancyplain} % Makes all pages in the document conform to the custom headers and footers
\fancyhead{} % No page header - if you want one, create it in the same way as the footers below
\fancyfoot[L]{} % Empty left footer
\fancyfoot[C]{} % Empty center footer
\fancyfoot[R]{\thepage} % Page numbering for right footer
\renewcommand{\headrulewidth}{0pt} % Remove header underlines
\renewcommand{\footrulewidth}{0pt} % Remove footer underlines
\setlength{\headheight}{13.6pt} % Customize the height of the header

\numberwithin{equation}{section} % Number equations within sections (i.e. 1.1, 1.2, 2.1, 2.2 instead of 1, 2, 3, 4)
\numberwithin{figure}{section} % Number figures within sections (i.e. 1.1, 1.2, 2.1, 2.2 instead of 1, 2, 3, 4)
\numberwithin{table}{section} % Number tables within sections (i.e. 1.1, 1.2, 2.1, 2.2 instead of 1, 2, 3, 4)

%\setlength\parindent{0pt} % Removes all indentation from paragraphs - comment this line for an assignment with lots of text

%----------------------------------------------------------------------------------------
%	TITLE SECTION
%----------------------------------------------------------------------------------------

\newcommand{\horrule}[1]{\rule{\linewidth}{#1}} % Create horizontal rule command with 1 argument of height

\title{	
\normalfont \normalsize 
\textsc{ISTIC - Département Informatique} \\ [25pt]
\horrule{0.5pt} \\[0.4cm]
\huge DM de Compilation \\[0.2cm]
\Large Prise en main de l'outil ANTLR
\horrule{2pt} \\[0.5cm]
}

\author{Bihel Simon - Daniel Lesly-Ann}

\date{\normalsize\today}

\begin{document}

\maketitle

%------------------------------------------------

\section*{Introduction}
\textsc{Antlr} est un outil permettant de générer des analyseurs lexicaux ou syntaxiques dans un langage cible, notamment en Java. Il peut être utilisé entre autre, pour construire un traducteur, un interpréteur ou un compilateur à partir de règles de grammaire définies par l'utilisateur. Il permet aussi de générer facilement des arbres syntaxiques, de gérer les erreurs et d'associer des actions aux règles de grammaire.



%------------------------------------------------

\section{Fonctionnement de l'outil \textsc{antlr}}

L'outil \textsc{Antlr} lit une grammaire en notation EBNF, qui a une syntaxe
proche des expressions régulières, et permet notamment l'utilisation des
symboles `*', `?' et `+' pour la définition des littéraux. Il produit ensuite un
analyseur syntaxique en Java que l'on peut utiliser dans un programme principal
pour analyser des données.

%Méthode d'analyse syntaxique : à détailler. Analyse LL(*) -> pas de 
%récursivité à gauche.
%http://www.antlr.org/papers/LL-star-PLDI11.pdf paper about the parser generator
\textsc{Antlr} utilise une stratégie d'analyse syntaxique \textit{top-down} 
appelée LL(*). Cette stratégie permet d'accepter tout type de grammaire mise à 
part la récursivité à gauche.

On peut différencier les règles de grammaire en deux types.
\begin{itemize}
 \item Les règles lexicales ne contiennent que des littéraux (qui peuvent 
 contenir des symboles EBNF) ou d'autres règles lexicales. Par convention dans 
 \textsc{Antlr} ces règles commencent par une majuscule.
 \item Les règles syntaxiques peuvent contenir des littéraux, des règles 
 lexicales ou des règles syntaxiques. Par convention dans \textsc{Antlr} ces 
 règles commencent par une minuscule.
\end{itemize}

% Section de remplissage si on a pas assez de contenu !

%------------------------------------------------

\section{Exemple : évaluateur d'expression}

L'exemple fourni permet d'évaluer des expressions mathématiques utilisant des 
identificateurs, les opérateurs `=', `+', `-', `*', ainsi que le parenthésage. 
Il permet aussi de construire un arbre de syntaxe abstraite pour chaque ligne 
de l'entrée.
[
- expressions
- identificateurs : stockage dans une hashmap
]


Comme illustré dans le listing \ref{expr_file}, on peut ajouter la division dans la règle de la multiplication, les deux opérations ayant la même priorité. L'associativité à gauche est respectée naturellement par \textsc{Antlr} puisqu'il utilise une analyse LL - de gauche à droite.

Dans le cas de la puissance, il est nécessaire d'ajouter une règle pour
respecter la priorité. Il faut aussi prendre en compte l'associativité à droite
en forçant l'analyseur à évaluer la partie à droite de l'opérateur. On peut
faire cela en utilisant l'opérateur `?' au lieu de `*' qu'on a utilisé
précédemment, comme on peut le voir dans le listing \ref{expr_file}. Cet
opérateur permet de n'avoir au maximum qu'une seule sous-règle, ce qui permet
donc de forcer par de la récursivité la création de l'arbre de la racine aux
feuilles.
% c'est pas clair

    
\begin{lstlisting}[label=expr_file,caption=Fichier Expr.g,
linebackgroundcolor={\lstcolorlines[orange!30]{8,9,10}\lstcolorlines[blue!30]{5}}]
expr:   multExpr (('+'^|'-'^) multExpr)*
    ;

multExpr
    :   powExpr (('*'^|'/'^) powExpr)*
    ;

powExpr
    :   atom ('^'^ powExpr)?
    ;

atom:   INT
    |   ID
    |   '(' expr ')'
    ;
\end{lstlisting}

Il faut ensuite ajouter les actions correspondantes pour la construction de l'arbre. Comme on peut le voir dans le listing \ref{eval_file}, chacun des opérateurs ajoute un nœud à l'arbre qui a pour valeur l'opération appliquée aux deux sous-arbres. Si l'expression est un identifiant, on récupère sa valeur dans la table de hachage et s'il n'est pas défini, on renvoie un message d'erreur. S'il s'agit d'un entier, on transforme la chaîne de caractère en l'entier correspondant. 

\begin{lstlisting}[label=eval_file,caption=Fichier Eval.g,
linebackgroundcolor={\lstcolorlines[orange!30]{6}\lstcolorlines[blue!30]{5}}]
expr returns [int value]
    : ^('+' a=expr b=expr)  { $value = a+b; }
    | ^('-' a=expr b=expr)  { $value = a-b; }
    | ^('*' a=expr b=expr)  { $value = a*b; }
    | ^('/' a=expr b=expr)  { $value = a/b; }
    | ^('^' a=expr b=expr)  { $value = (int) Math.pow(a,b); }
    | ID
      {
        Integer v = (Integer) memory.get($ID.text);
        if ( v != null ) $value = v.intValue();
	else System.err.println("undefined variable "+$ID.text);
      }
    | INT { $value = Integer.parseInt($INT.text); }
    ;
\end{lstlisting}



%------------------------------------------------

\section{Tests}

Le fichier d'exécution fournit permet la visualisation de l'arbre syntaxique de
d'une expression fournie. On peut donc évaluer notre grammaire sur la
concordance des opérateurs lors de l'évaluation, la priorité des opérations et
leurs associativités. Étant donné le faible nombre d'opérateurs on peut juste
vérifier à la main leur validité sans passer par un générateur qui rajouterait
une couche potentielle d'erreurs. Pour la priorité des opérations il suffit de
prendre une expression avec tous les opérateurs et vérifier que l'arbre est
correct. Il faut ensuite permuter les opérations pour être sûr que l'ordre de
lecture n'a aucun impact, puis vérifier que les parenthèses sont bien prises en
compte. L'associativité se vérifie en regardant l'encapsulation des parenthèses.
Pour finir quelque tests aléatoires peuvent être effectués en faisant attention
aux problèmes d'overflow.
%je liste juste les problèmes/bugs qui pourraient arriver car je pense que
%faire un générateur de tests pour des opérateurs atomiques ça rajoute juste
%une couche d'incertitude (?)



\end{document}


%------------------------------------------------

\section*{Conclusion}

%------------------------------------------------


\end{document}